%%%%%%%%%%%%%%%%%%%%%%%%%%%%%%%%%%%%%%%%%%%%%%%%%%%%%%%%%%%%%%%%%%%%%%%%%%%%%%%%%%%%%%%%%%%%%%%%%%%%%%%%%%%%%%%%%%%%%%%%

%%%%%%%%%%%%%%%%%%%%%%%%%%%%%%
% FRONT MATTER
%%%%%%%%%%%%%%%%%%%%%%%%%%%%%%

\begin{frontmatter}

\maketitle

%%%%%%%%%%%%%%%%%%%%%%%%%%%%%%%%%%%%%%%%%%%%%%%%%%%%%%%%%%%%%%%%%%%%%%%%%%%%%%%%%%%%%%%%%%

%%%%%%%%%%%%%%%%%%%%%%%%%%%%%%
% ABSTRACT
%%%%%%%%%%%%%%%%%%%%%%%%%%%%%%

\begin{abstract}\label{sec:abstract}

Critical infrastructure systems form the foundation for the economic prosperity, security, and public health of the modern world. These complex, interdependent systems are prone to failures from causes such as natural hazards (\emph{e.g.}, hurricanes), terrorism, and deterioration of aging components, which can result in severe disruptions to critical services provided to society.  Therefore, to minimize threats to society posed by failures in infrastructure systems, it is important to conduct risk and reliability analyses to identify and address system vulnerabilities.  However, the large geographic scale and the high degree of complexity within and between infrastructure systems pose significant challenges for modeling the performance and reliability of infrastructure systems. Thus, this dissertation addresses deficiencies in current methods for modeling infrastructure system reliability by developing approaches that reflect physical and engineering details governing network performance, yet are also scalable to complex systems covering large geographic areas.

The objectives of this work are achieved through the completion of three projects. Chapter \ref{ch2} examines the relationship between network topology and network robustness to random failures and targeted attacks for randomly generated networks. I demonstrate that there is a statistically significant relationship between the initial topological properties of scale-free networks and their corresponding robustness to both random failures and targeted attacks. I also use this statistical approach to accurately estimate network robustness to failures for real-world networks.

Chapter \ref{ch3} compares topological and physical performance models for quantifying performance of electric power networks.  I present a classification for different types of functional models that can be used for risk and vulnerability analysis of electric power systems, and compare the estimates of system performance obtained with these models to an AC power flow model.  I show that in general, the greater the inclusion of physical characteristics of the system in a functional model, the better the estimate of the system’s actual performance when perturbed.  Additionally, I demonstrate that statistical models combining simplified topological measures can be used as a surrogate for physical flow models for predicting electric power system performance after failures.

Finally, Chapter \ref{ch4} applies an approach for modeling ecological networks to modeling interdependent infrastructure systems.  Here, I demonstrate the use of `Muir webs' for capturing additional dependencies within and between infrastructure systems (\emph{e.g.}, power supply to pumps in water systems) and management factors (\emph{e.g.}, availability of operators).  I show that the Muir web approach provides the basis for a more realistic representation and estimation of the performance and reliability of interdependent infrastructure systems.

The work presented in this dissertation represents a significant contribution to the field of infrastructure risk and reliability analysis.  The relative simplicity of the models developed here, both in required data and in computational complexity, makes them a highly practical and efficient tool for aiding real-world decision-making.  And, incorporating important physical and engineering details of infrastructure system behavior ensures that the guidance they provide to decision-makers allows for optimal improvements to system reliability.


\vspace{1cm}

\noindent Advisor: Dr. Seth Guikema\\
Committee: Dr. Ben Hobbs and Dr. Jonas Johansson

\end{abstract}

%%%%%%%%%%%%%%%%%%%%%%%%%%%%%%%%%%%%%%%%%%%%%%%%%%%%%%%%%%%%%%%%%%%%%%%%%%%%%%%%%%%%%%%%%%

%%%%%%%%%%%%%%%%%%%%%%%%%%%%%%
% ACKNOWLEDGMENTS
%%%%%%%%%%%%%%%%%%%%%%%%%%%%%%

\begin{acknowledgment}\label{sec:acknowledgments}

First and foremost, I would like to thank my adviser, Dr. Seth Guikema, for his unwaivering support throughout the course of my PhD work.  His continued encouragement and enthusiasm for research has been instrumental in allowing me to complete my own research.  He has taught me a tremendous amount, both in subject matter and in what it takes to be a good researcher. I am forever grateful for the countless opportunities he has provided me with to broaden my research horizons. I could not have asked for a better adviser.

I would like to thank Dr. Ben Hobbs for always providing insightful questions and comments on my work. His breadth of knowledge and passion for new ideas has been inspiring in my research journey.

I would also like to thank Dr. Jonas Johansson, Dr. Henrik Hassel, and Dr. Kurt Petersen for allowing me to visit Lund University.  I am very glad to have had the opportunity to collaborate with Jonas and Henrik, and am appreciative of their contributions to the paper we wrote together.

Finally, I would like to thank the members of my research group, particularly Roshi Nateghi and Andrea Staid, for their tremendous support and encouragement during my years at Johns Hopkins.

\end{acknowledgment}


%%%%%%%%%%%%%%%%%%%%%%%%%%%%%%%%%%%%%%%%%%%%%%%%%%%%%%%%%%%%%%%%%%%%%%%%%%%%%%%%%%%%%%%%%%

%%%%%%%%%%%%%%%%%%%%%%%%%%%%%%
% TABLE OF CONTENTS
%%%%%%%%%%%%%%%%%%%%%%%%%%%%%%

% generate table of contents
\tableofcontents

% generate list of tables
\listoftables

% generate list of figures
\listoffigures

%%%%%%%%%%%%%%%%%%%%%%%%%%%%%%%%%%%%%%%%%%%%%%%%%%%%%%%%%%%%%%%%%%%%%%%%%%%%%%%%%%%%%%%%%%

\end{frontmatter}

%%%%%%%%%%%%%%%%%%%%%%%%%%%%%%%%%%%%%%%%%%%%%%%%%%%%%%%%%%%%%%%%%%%%%%%%%%%%%%%%%%%%%%%%%%%%%%%%%%%%%%%%%%%%%%%%%%%%%%%%
